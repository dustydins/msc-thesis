% Chapter 1

\chapter{Introduction} % Main chapter title
\label{Chapter1} % For referencing the chapter elsewhere, use \ref{Chapter1} 

\lhead{Chapter 1. \emph{Introduction}} % This is for the header on each page - perhaps a shortened title

%----------------------------------------------------------------------------------------
% Section - Context
%----------------------------------------------------------------------------------------

\section{Context}

%-- machine learning in safety-critical systems
\Gls{ml} algorithms are becoming increasingly present in systems that operate within shared environments
with humans, or involve direct interaction with humans themselves~\citep{pereira}. These systems 
are often defined as safety-critical, such that their failures lead to unintended and potentially harmful behaviours~\citep{amodei}.
Examples of these systems include autonomous automotive systems, traffic control systems, medical devices, aviation software,
industrial robotics, and many more cyber-physical systems that interact with our environment.
Many of these systems have so far only existed as proof of concepts, but are steadily approaching commercial use within our society.

%-- prone to adversarial attacks
Additionally, recent research has exposed broad vulnerabilities to adversarial attacks within data driven \gls{ml} algorithms,
including \Glspl{nn}; where applying small but intentional perturbations to an input which are not noticible to humans,
can lead to a model outputting an incorrect classification with high confidence~\citep{goodfellow}.
An example of such an attack can be seen in \textit{Fig.~\ref{fig:adversarialpatch}}.
Consequently, the testing and verification of \gls{ml} for the use of controlling safety-critical systems has become a focused area of research in recent years.

\begin{figure}[H]
	\centering
        \includegraphics[width=0.8\textwidth]{media/introduction/sticker.png}
        \rule{35em}{0.5pt}
        \caption[Google's Aversarial Patch]{\textbf{Google's Adversarial Patch} -- An example of a method to create targeted adversarial attacks on \glspl{nn} by adding carefully designed noise via a physical patch~\citep{brown2018}.}\label{fig:adversarialpatch}
\end{figure}

%-- definition of software testing and verification
This thesis will use the following definitions for software testing and verification.
Software testing, or validation, is defined as the evaluation of a system under various conditions and observing its behaviour while
looking for defects~\citep{pereira}. In the context of \gls{ml} development, testing is used to
ensure that a trained model generalises accurately to some previously unseen test data.

Verification is defined as the process of determining whether the products of a phase of the software development process fulfill
the requirements established during the previous phase~\citep{ammann2008}. Formal verification in other words, formulates logical arguments
that a system will not act abnormally under a wide range of circumstances, and can be used to determine not only generality, but also the robustness and correctness of a system.

%-- challenges when verifying ML systems
The challenges regarding verification of \gls{ml} models stem from the typically less deterministic and more statistically-oriented nature of their algorithms, which
lead to a lower degree of understanding than software that is explicitly programmed to perform a specific task~\citep{bishop}. These types of systems are
commonly referred to as \textit{black box} systems, where the internal mechanisms are not revealed; in other words, it is impossible to understand a model just by
looking at its parameters~\citep{molnar2019}.

%----------------------------------------------------------------------------------------
% Section - Motivation
%----------------------------------------------------------------------------------------
\section{Motivation}

Public calls for \textit{sensible} or \textit{verifiable \Gls{ai}} have been raised in recent years due to ever increasing
development of complex and pervasive systems that are entering into our everyday lives~\citep{russell2016}.

%-- current trends
%-- TODO | change references to current verification systems
Formal verification of deterministic software systems has seen significant progress since the early
verification systems developed in languages such as Lisp, Ada, and Pascal~\citep{polak1979, boyer1990, guaspari1993}. However,
verification of non-deterministic systems has seen relatively little progress, with the exception of
\Glspl{mas}~\citep{lomuscio2017, kouvaros2016}.

%-- different languages often used for ml, than embedded/sys programming where formal verification widely used
Indeed, due to the contemporary nature of \gls{aiv} research, there are limited resources with regard to the 
programming tools available for researchers in this area. This is especially true for work
within \gls{ml}, as the programming languages and tools commonly used for \textit{traditional} verifcation are often
disparate from those widely adopted by the \gls{ml} communities. 
% Many libraries such as Coq and Z3 need additional bindings developed to work in these languages

%-- ml languages
Popular programming languages used for \gls{ml} such as Python or Matlab currently have comparitively less
formal verification tools available than those concerned with system infrastructure or embedded applications.
Additionally, \gls{aiv} toolkits for \gls{ml} tasks in these languages are still in early stages of development, and mainly focused
on the verification of \glspl{nn}~\citep{kokke2020}. 

%-- programming languages are forever changing
Furthermore, the landscape of \gls{ml} programming itself is forever shifting, and while there is yet a programming 
language dedicated for \gls{ml} tasks, huge efforts from programming language designers have been made 
in developing \gls{ml} libraries for existing languages. This is necessary in order to handle the 
extremely high computational demands, and to simplify 
model languages to make them easier to add domain-specific optimisations and features~\citep{innes2017}.

%- GO

A prime example of such development can be seen in the Go programming language, or \textit{GoLang}. A relatively new language, originally developed by Google in 2009 with the intention
of creating a modern general-purpose language similar to C. 
GoLang has seen a surge in popularity within the \gls{ml} community since
the release of its first extensive \gls{ml} package, \textit{Gorgonia}, in 2016, which heavily relies on the use of expression graphs~\citep{chew2016}.
This package allows GoLang developers
to take advantage of automatic and symbolic differentiation, gradient descent optimisations, numerical stabilisation,
added support for CUDA/GPGPU computation, and comparitively quick speeds than its Python counterparts (Theano and TensorFlow)~\citep{golang2020}.

% -- TODO | discuss Microsoft's efforts in F (as annotated by Katya)

%-- need for aiv development in other languages
Consequently, as programming languages continue to develop \gls{ml} capabilities, there is a need for 
exploring new and scalable approaches for developing \gls{aiv} tools in these languages.
This is especially important for programming languages which are being adopted by industry to implement \gls{ml} models
for the use within safety-critical or pervasive systems.

%----------------------------------------------------------------------------------------
% Section 3
%----------------------------------------------------------------------------------------
\section{Aims \& Objectives}

%-- aims
% explore the methods for efficiently implementing formal verification tools in a wide range of
% programming languages used for machine learning

% Investigate different approaches for implementing formal verification tools in a wide range of programming
% languages used for \gls{ml}

%-- talk about go, and machine learning

%- Go safety critical applications

%- aim

The aim of this project is to investigate the current programming paradigms within \gls{ml} development,
and to explore the suitability of current formal verification toolkits available to them. Subsequently,
this thesis will aim to design and implement a GoLang formal methods framework for Gorgonia \glspl{nn},
providing GoLang \gls{ml} developers with a set of tools which will allow them to produce safe and
fair \gls{ai} applications.

This framework will extend upon the work made by~\citep{kokke2020}, and the Sapphire library implemented in Python
which successfully translates TensorFlow feed-forward \gls{nn} models to the Z3 \Gls{smt} solver created by Microsoft Research~\citep{demoura2008}.

% - objective

% To achieve this, the primary objective is to implement a GoLang package which allows the binding of Gorgonia \gls{nn} models to
% Z3 SMT solvers. Similar to the Sapphire library written in Python~\citep{kokke2020}.

To achieve this project's aims, the following objectives should be met:

\begin{itemize}
    \setlength\itemsep{0em}
    \item \textit{Objective 1 -} Conduct a feasibility study with regards to developing a formal methods framework for \glspl{nn} in Go.
    \item \textit{Objective 2 -} Implement bindings that map the parameters of a Gorgonia \gls{nn} model to
        Z3 variables.
    \item \textit{Objective 3 -} Select data in order to train and verify \gls{nn} models using this project's formal methods framework.
    \item \textit{Objective 4 -} Implement a series of \gls{nn} models in Gorgonia using the data sets mentioned in \textit{Objective 3}.
    \item \textit{Objective 5 -} Verify the correctness of Gorgonia \glspl{nn} using the bindings mentioned in \textit{Objective 2}.
    \item \textit{Objective 6 -} Make conclusions about the developed framework's benefits and limitations, and discuss future improvements to the methodology as described in \textit{Objective 1}.
\end{itemize}
